\documentclass[11pt, oneside]{article} 
\usepackage{geometry}
\geometry{letterpaper} 
\usepackage{graphicx}
	
\usepackage{amssymb}
\usepackage{amsmath}
\usepackage{parskip}
\usepackage{color}
\usepackage{hyperref}

\graphicspath{{/Users/telliott/Dropbox/Github-math/figures/}}
% \begin{center} \includegraphics [scale=0.4] {gauss3.png} \end{center}

\title{Pizza theorem}
\date{}

\begin{document}
\maketitle
\Large

%[my-super-duper-separator]

I found this problem, called the ``pizza theorem", in Acheson's \emph{The Wonder Book of Geometry}. 

Consider a circular pizza pie.  Choose a point \emph{anywhere} in the pie.  Choose any orientation and draw one pair of perpendicular chords crossing at the point.  Fill in with another pair of perpendicular chords on the same center, but rotated ($45^{\circ}$).
\begin{center} \includegraphics [scale=0.65] {Acheson_G111.png} \end{center}

Now form the sum of the areas of alternate slices.  Above, the two collections are shaded to tell them apart.  The total dark area and the total light area are always equal.  The pizza is evenly divided, even though there's no obvious symmetry and the slices are wonky.

\url{https://en.wikipedia.org/wiki/Pizza_theorem}

\subsection*{setup}

We will refer to the point where all the chords cross as the \emph{grid center}.  It is clear that if the grid center coincides with the circle center, then by the radial symmetry of the circle and the equal radial angles, all the segments will be the same.  The equal area result is obvious.
\begin{center} \includegraphics [scale=0.3] {pizza10.png} \end{center}

For these drawings, I've chosen to mark the ``shaded" slices with dots.  

Now, slide the grid center away from the circle center along a diagonal of the grid, which at this point is also a diagonal of the circle.  Because the circle is featureless, we can pick any diagonal, without loss of generality.

In the right panel below, the grid center has moved vertically down from the circle center, which is marked with a magenta dot.
\begin{center} \includegraphics [scale=0.3] {pizza11.png} \end{center}
The equal area result still holds.  The reason is that the figure has mirror image symmetry across the vertical line, and because of the alternation of light and dark, each pair of mirror-image shapes has one of each type.

\subsection*{preliminary analysis}

We might try to analyze a general location for the grid center, not on a diagonal, and ask whether the areas add up properly.  In fact, that is the idea of a famous proof without words, which we'll show later.

However, we will do something different and instead analyze the movement of the grid center away from the circle center.  We can show that each change leaves the allocation of area between light and dark parts unchanged.  Since we start with equal areas and the movements don't change area, the proof will be complete.
\begin{center} \includegraphics [scale=0.4] {pizza4.png} \end{center}

There are two approaches to the second movement, after a preliminary movement along a diagonal of the grid (and circle).  As we said, that does not change the relative areas.  One views the image radially, so having reached the desired distance from the circle center, \emph{rotate} about the grid center to achieve the desired final orientation.

Rotation can only be performed as the second operation, since otherwise the grid center will remain on a diagonal of the circle.

There is a second possibility, to move \emph{horizontally} in the second phase.  The grid center still runs runs parallel to a diagonal of the grid, but this is no longer on a diagonal of the circle.

The question arises whether this second method can actually achieve \emph{any} position and orientation.

But consider the reverse movement.  Start from any random position and orientation.  Rotate the paper (or the circle) so that one chord is horizontal.  This hasn't changed the figure and certainly doesn't change the relative areas.  The picture we gave above with one chord horizontal, is still accurate.  

Now, move along a line of the grid until the grid center reaches the diagonal of the circle perpendicular to the direction of movement.  We have achieved equal areas.  

So the crux of the analysis is that preliminary movement along a diagonal, followed by either radial or horizontal motion, can achieve an arbitrary position.  We deal with horizontal motion first, since that is the part of the problem that I actually solved myself.  We just note in passing that we need only move the chord center as far as the position where the chord at $45^{\circ}$ to the vertical was originally a diagonal of the circle.  Only positions in one sector of the pie matter, but that's good enough.

\begin{center} \includegraphics [scale=0.35] {pizza_sector.png} \end{center}

\subsection*{horizontal movement}

\begin{center} \includegraphics [scale=0.35] {pizza13.png} \end{center}
Focus on a particular chord with arms $t$ and $s$ (one of the four chords is not shown).  Movement of the grid center is to the right, magenta arrow.  

The circle center is in red, and the second black dot marks the center of the chord.  The chord center is the point where the distance to the edge of the circle is equal in both directions.

We make the following observations:

$\circ$ \ The chord center is constrained to fall on the red line, its perpendicular bisector.  Crucially, the bisector never changes direction, only the position of the chord center on it, i.e. the distance to the circle center.

That's because in this movement the grid center doesn't rotate, the chord never changes its angle, and the final position is parallel to the beginning position.

$\circ$ \ The distance from chord center to circle center changes by the amount of the horizontal movement, with an extra factor of $\sqrt{2}$ imposed by the geometry.

$\circ$ \ If the region just above and left of arm $t$ is ``shaded", then that below and right of arm $s$ is also shaded.  The net change in shaded area by movement of \emph{this} chord is $t$ \emph{minus} $s$, multiplied by a factor we will show next.

$\circ$ \ The curved shapes formed by the movement at the ends of any chord are mirror images.  Since one increases and the other decreases shaded area, the effects cancel.  We might call these small curved shapes \emph{wingtips}.

That last point may be appreciated from the diagram below, which shows movement to the right.  Because the starting and ending positions of the chords are parallel, the wingtips are exactly the same size and of opposite shading.
\begin{center} \includegraphics [scale=0.35] {pizza4b.png} \end{center}

Consider the total increase in shaded area due to this movement.  The increase comes from rectangles formed by arms of three chords:  $c$, $b$ and $h$.  The decrease in shaded area is due to rectangles on the other end of the same chords:  $a$, $d$ and $g$.  The horizontal chord makes no contribution.

The net change in area for a horizontal change $\Delta x$ is
\[ \Delta A = \ [ \ \sqrt{2}(b - a) + (c - d) + (h - g) \ ] \ \Delta x \]
The chord $ab$ counts more because it is perpendicular to the direction of travel, so it captures more area.

Our hypothesis is that $\Delta A = 0$ so we expect to find that
\[ \sqrt{2}(a - b) = (c - d) + (h - g) \]

\subsection*{solution}
The core of the argument is to connect the equation above for the change in area to something else, namely, the distances of chord centers from the grid center.  

Consider chord $gh$, with arms $g$ and $h$.  Given the assignment of shaded areas above, the change in shaded area as the chord moves to the right is $(h-g) \ \Delta x$.

But this term $(h-g)$ is also related to the distance from the chord center of $gh$ to the grid center.  $(h-g)$ is the length of a line segment centered on the chord center, where one end is on the grid center.  So $(h-g)/2$ is the actual distance from the chord center of $gh$ to the grid center.  (If this isn't clear you might review \hyperref[sec:rectangle_side_on_a_circle]{\textbf{rectangles in circles}}).
 
But because of all the right angles, we have rectangles, so that is also exactly the same as the distance of the center of chord $cd$ from the center of the circle.
\begin{center} \includegraphics [scale=0.4] {pizza7.png} \end{center}

We can show that the net change in the latter distance for the chords $cd$ and $gh$, considered together, is zero.  Therefore the change in shaded area is also zero.

\emph{Proof}.
Consider an isosceles right triangle with sides scaled to have unit length and hypotenuse equal to $\sqrt{2}$.  Pick any point along the base that is length $a$ from the left side, and draw a rectangle (red).  
\begin{center} \includegraphics [scale=0.4] {pizza15.png} \end{center}
The perimeter of the rectangle has length $2(a + b) = 2$, no matter which point is chosen, because by similar triangles the smaller triangles are also isosceles.

In the same way (right panel), when we slide the grid center along the horizontal grid line, the centers of the two chords at $45^{\circ}$ are constrained to lie along the sides of an isosceles triangle (red lines).

The rectangle that is formed has a constant perimeter.  The sum of distances from the center of the circle to the two chord centers is one-half that, so also constant.  We can even calculate the distance as $\sqrt{2}$ times the vertical distance from the circle center to the horizontal chord.

$\square$

Call the distance from the center of a chord to the grid center $\delta$.  Take the last equation above, and divide both sides by $2$.  This gives an \emph{invariant} which obviously holds when the grid center remains on a diagonal of the circle (vertical movement).  In fact, by symmetry $\delta_{cd} = \delta_{gh}$ so 
\[ \delta_{ab} = \sqrt{2} \ \delta_{cd} \]

The red lines go from the center of each chord to the center of the circle (red dot).  These are perpendicular bisectors of the angled chords.  Because of the right angles, each one is also equal to $\delta$ for the perpendicular chord.

The vertical down from the red dot (center of the circle) to the grid center is $\delta_{ab}$, since the center of this chord coincides with the center of the circle.

\begin{center} \includegraphics [scale=0.4] {pizza7.png} \end{center}
It is clear that for the central position $\delta_{ab} = \sqrt{2} \ \delta_{cd}$, as we said above.  $\delta_{ab}$ is the diagonal of a square with sides of length $\delta_{cd} = \delta_{gh}$.

Next, consider two positions reached by horizontal movement:
\begin{center} \includegraphics [scale=0.4] {pizza12.png} \end{center}

These movements don't change $\delta_{ab}$ but they \emph{do} change $\delta_{cd}$ and $\delta_{gh}$.  The latter two are adjacent sides in a rectangle whose four vertices are the two chord centers, plus the circle center and the grid center.  It is a rectangle because the two chords are perpendicular and the red lines are perpendicular bisectors.

Both chords have successive positions lying along fixed lines connected to the circle center (red lines).  The distance each moves for any given horizontal translation is the same (with opposite sign), because each center moves along the hypotenuse of an isosceles right triangle, whose base has the magnitude of the movement.

Thus, the total distance from the two chord centers to the center of the circle does not change for horizontal movement.  Because of the rectangular arrangement, this distance is equal to $\delta_{cd} + \delta_{gh}$, so the invariant doesn't change either.

Recall that we need only consider the movement as far as the point where chord $cd$ becomes a diagonal of the circle (see above).  This is also the point at which $\delta_{cd}$ goes to zero.

\subsection*{summary}

We have shown that 
\[ \sqrt{2} \cdot \delta_{ab} =  \delta_{cd} + \delta_{gh} \]

holds regardless of motion horizontally.

In fact neither the left-hand nor the right-hand side shows a net change at all.  It follows that the areas we have discussed as shaded and unshaded are also invariant under horizontal translation, even when it's not along a diagonal of the circle.

The starting position --- grid center at the center of the circle --- has the property of equal areas for shaded and unshaded regions, and the transformations do not change the invariant.  Therefore, the ending position --- grid center anywhere in the circle and with any orientation --- also has the property of equal areas.

$\square$

The standard proofs of this theorem analyze rotation.  I found an approachable one for rotation that involves only trigonometry, here:

\url{https://math.stackexchange.com/questions/865818/can-anyone-explain-pizza- theorem}

We will need a special preliminary result.

\subsection*{extraordinary property}

Consider two chords at right angles in a circle.  Form the rectangle using a side equal and parallel to chord $ab$.  (Only two sides of the rectangle are shown).  An easy way to do this is to draw the two diagonals (dotted lines).
\begin{center} \includegraphics [scale=0.35] {perp_chords1.png} 
\includegraphics [scale=0.35] {perp_chords3.png} 
\end{center}

From a standard result about rectangles in circles, referred to above, we can conclude that the extension at the top has length $c$, so the side of the rectangle is $d-c$.  Applying the Pythagorean theorem
\[ (a + b)^2 + (c - d)^2 = (2R)^2 \]
\[ a^2 + 2ab + b^2 + c^2 - 2cd + d^2 = (2R)^2 \]
but $ab = cd$ by the crossed chords theorem so
\[ a^2 + b^2 + c^2 + d^2 = (2R)^2 \]
When the components of two chords crossed at right angles in a circle are squared and summed, the result is twice the radius, squared.  This is called the ``extraordinary property" of the circle.

\subsection*{area of a sector}

\begin{center} \includegraphics [scale=0.4] {pizza6.png} \end{center}

Rotation is connected to the special property of the circle.  Consider a counter-clockwise rotation.

For a very small angle $\Delta \theta$, the area of a sector swept out by that angle is 
\[ A = \frac{1}{2} \ r \Delta \theta \cdot r = \Delta \theta \ \frac{r^2}{2} \]

If we focus on the sectors marked with red dots, a counter-clockwise rotation adds to the area a small wedge on the left arm, viewed from the central point and facing out.  It will be seen that the relevant radii are all at right angles to one another.

These values of $r$ are exactly those which were involved in the extraordinary property:  one has a factor of $\Delta \theta/2$ times $a^2$, another has $b^2$, the next $c^2$ and finally $d^2$ (using the notation from the special property theorem).

Therefore, the total increase in area is
\[ \Delta A = \Delta \theta \ \frac{a^2 + b^2 + c^2 + d^2}{2} \]
\[ = \Delta \theta \ \frac{4R^2}{2} = \Delta \theta \ 2R^2 \]

The increase in area for four alternate segments with a small rotation depends on $\Delta \theta$ with a constant multiplier related to $R$ for the circle (not $r$ for the sector).

Furthermore, the shaded regions lose exactly the same amount of area at the trailing edge, along the right arms of the sectors.

The unshaded areas, although they have different radii, add and subtract exactly the same areas, and for the same reason.

A subtlety that is often neglected is that the result is only plausible and not rigorous, because we have assumed that $r$ doesn't change \emph{significantly} with small $\Delta \theta$.  

I am not 100\% happy with this geometric argument, because of the oddly shaped regions at the ends of the arms.  The rate of change of $r$ with $\Delta \theta$ is different for the different $\theta_i$.  For that reason, we'll say a little more.

One observation that might help is to notice that each angle of $\pi/4$ at the intersection is equal to the average of the two whole arcs swept out by it and its companion vertical angle.  I haven't worked through the implications yet, it just seems possibly useful.  There may be a simple proof in there somewhere, since it turns out that the net effect on all the $\theta_i$ together is zero.

\subsection*{calculus}

We might use calculus for this.

\url{https://www.math.uni-bielefeld.de/~sillke/PUZZLES/pizza-theorem}

In that language we would have that (twice) the area added by a rotation is

\[ \sum_i r_i^2 (\phi_i) \cdot \Delta \phi = \sum_i  \int_{\theta_1}^{\theta_2} r_i^2 (\phi_i) \ d \phi \]
where $r_i(\phi_i)$ says that each $r$ is a (complicated) function of the angle $\phi$, for a chosen central point.

But the sum of the integrals is the integral of the sum.  

\[ = \int_{\theta_1}^{\theta_2}  \sum_i  r_i^2 (\phi_i) \ d \phi \]

This is called Fubini's theorem.

\url{https://en.wikipedia.org/wiki/Fubini%27s_theorem}

And when we add them together, the sum of those factors $r^2$ is a constant, by the extraordinary property of the circle.  So in the end we obtain $(2R)^2$ times the total of the four angular rotations.  The changes in the radius simply drop out from the calculation.

I am not entirely happy with this argument either.  As before, the reason is that the oddly shaped regions at the ends of the arcs are different for the different chords.  In other words, the functions $r(\theta_i)^2$ must be \emph{different} for the various $i$.

\subsection*{resolution}

I finally found a solution I am happy with on the web.  It shows that when you work through the algebra, almost everything really does cancel, and what's left is constant.  It does use a coordinate system and polar coordinates but no real calculus.

\url{https://math.stackexchange.com/questions/865818/can-anyone-explain-pizza-theorem}

(from the answer marked as correct with a green checkmark).

We will write equations for each area that must be integrated over, each $r(\theta_i)^2$ (i.e. $[r(\theta_i)]^2$).  These functions are \emph{different} but closely related.  

The integrals are polar integrals.  Although they are full of sines and cosines, the $\theta_i$ are at right angles to each other.  The integral will turn out to be trivial, because the great simplification referred to above, turns out to be correct.

Place the origin of coordinates at the grid center, the point where all the chords cross.  Let the coordinates of the circle's center be $(x, y)$.

Let $\theta$ be the angle that the radius of the arc $r$ makes with the horizontal.  The coordinates of any point where $r$ meets the circle are $( r \cos \theta, r \sin \theta)$.  

The basic relation of the circle is that the distance of that point on the circle from the center of the circle is $R$.  

The squared distance is
\[ R^2 = (r \cos \theta - x)^2 + (r \sin \theta - y)^2 \]
\[ = r^2 (\cos^2 \theta + \sin^2 \theta) - 2r x \cos \theta - 2 ry \sin \theta + x^2 + y^2 \]
\[ = r^2 - 2r (x \cos \theta + y \sin \theta) + x^2 + y^2 \]

This is a quadratic in $r$.  Rearrange to standard form.

We have some algebra ahead, so to make the notation simpler, let
\[ P(\theta) = x \cos \theta + y \sin \theta \]
In fact, let's call it $P$ and just remember that it is a function of $\theta$.

\[ r^2 - 2Pr + x^2 + y^2 - R^2 = 0 \]
Also
\[ Q = R^2 - x^2 - y^2 \]
$Q$ is a constant for any given position of the grid center.  So now

The solutions are
\[ r = \frac{1}{2} \cdot \ [ \ 2P \pm \sqrt{4P^2 - 4Q} \ ] \]
\[ = P \pm \sqrt{P^2 + Q} \]

Since $R^2 > x^2 + y^2$ we have that $Q > 0$, so the second solution (i.e. minus the square root of the determinant) is $< 0$, and it can be ignored since we're dealing with lengths.

\[ r(\theta) = P + \sqrt{P^2 + Q} \]
\[ r(\theta)^2 = P^2 + 2 P \sqrt{P^2 + Q} + P^2 + Q \]
\[ = 2P^2 + 2 P \sqrt{P^2 + Q} + Q \]

The tricky part is to figure out what these values for $r^2$ are for different $\theta$.  Luckily, we only have four angles to worry about, namely $\theta$ plus one each of  $\{-\pi/2, 0, \pi/2, \pi\}$.

\subsection*{one pair of values for $\theta$}

Start with $\theta$ and $\theta + \pi$.  When $\pi$ is added to the angle, both the cosine and the sine switch \emph{sign}.  So
\[ P(\theta + \pi) = - (x \cos \theta + y \sin \theta) = - P(\theta) \]

What's under the square root has $P^2$ and something that doesn't depend on $\theta$ so the only change for $r$ is the sign of the first term.
\[ r(\theta + \pi) = -P + \sqrt{P^2 + Q} \]
But that changes the sign of the mixed term in the square:
\[  r(\theta + \pi)^2 = 2 P^2 - 2 P \sqrt{P^2 + Q} + Q \]

When we add them that pesky square root disappears:
\[ r(\theta)^2 + r(\theta + \pi)^2 = 4P^2 + 2Q  \] 
\[ = 4 (x \cos \theta + y \sin \theta)^2 + 2Q \]
Pull out a factor of $2$ twice
\[ = 2 \ [ \ 2(x^2 \cos^2 \theta + 2xy \cos \theta \sin \theta + y^2 \sin^2 \theta) + Q \ ] \]

\subsection*{two more values for $\theta$}

Adding $\pi/2$ to $\theta$, sine and cosine switch places and also the \emph{sign} of $\cos(\theta + \pi/2)$ becomes negative.
\[ P(\theta + \pi/2) = x \cos (\theta + \pi/2) + y \sin (\theta + \pi/2) \] 
\[ = - x \sin \theta + y \cos \theta \]
On the other hand, subtracting $\pi/2$ gives
\[ P(\theta - \pi/2) = x \cos (\theta - \pi/2) + y \sin (\theta - \pi/2) \] 
\[ = x \sin \theta - y \cos \theta \]
So minus the first is equal to the second, as before.

Now we need to compute $2P^2 + 2P \sqrt{P^2 + Q} + Q$.  

We will not bother with the middle term.  As before, the change of sign on $P$ makes this disappear when we add them.  We need only $2P^2 + Q$ for each $\theta$.

The two terms $r(\theta \pm \pi/2)^2$ added together give
\[ 2 (- x \sin \theta + y \cos \theta)^2 + Q + 2 (x \sin \theta - y \cos \theta)^2 + Q \]

Using simpler variables to illustrate, this is (partly)
\[ (-a + b)^2 + (a - b)^2 = 2(a^2 - 2ab + b^2) \]
 times another factor of $2$:
\[ 2 \ [ \ 2(x^2 \sin^2 \theta - 2xy \sin \theta \cos \theta + y^2 \cos^2 \theta) + Q \ ]  \]

Now, go back to what we had before for $r(\theta)^2 + r(\theta + \pi)^2$
\[ 2 \ [ \ 2(x^2 \cos^2 \theta + 2xy \cos \theta \sin \theta + y^2 \sin^2 \theta) + Q \ ] \]

There are two more cancellations coming when we add.  First, the mixed term disappears.  Then also there is $\sin^2 + \cos^2$ so that leaves just
\[ = 2 \  [ \ 2(x^2 + y^2) + 2Q \ ] \]
This is the sum for all four angles:
\[ \sum_{k=0}^3 r(\theta + k \pi/2)^2 = 4x^2 + 4y^2 + 4Q \]
\[ = 4x^2 + 4y^2 + 4(R^2 - x^2 - y^2) \]
\[ = 4 R^2 \]

The sum of the integrands is independent of $\theta$ and hence invariant as $\theta$ changes.  

What a beautiful simplification!  That is really special.

\subsection*{references}

I found a number of images for a ``proof without words" online and I also found a reference to some related proofs, as well as a discussion on Stack Exchange:

\url{https://math.stackexchange.com/questions/865818/can-anyone-explain-pizza-theorem}

The reference says

\begin{quote}Sliding these arcs and
chords together, we see that the chords form a right triangle with the
diameter of the circle as the hypotenuse.\end{quote}

The various pictures of the proof without words do not explain how the shapes were arrived at.  However, they do show that the complementary light and dark areas each add up to one-half of the pizza.  

The original article is paywalled and the price is truly exorbitant.  You have to wonder if they've ever sold even a single copy for \$55.

\url{https://www.tandfonline.com/doi/abs/10.1080/0025570X.1994.11996228}

\begin{center} \includegraphics [scale=0.4] {pizza2.png} \end{center}


\end{document}