\documentclass[11pt, oneside]{article} 
\usepackage{geometry}
\geometry{letterpaper} 
\usepackage{graphicx}
	
\usepackage{amssymb}
\usepackage{amsmath}
\usepackage{parskip}
\usepackage{color}
\usepackage{hyperref}

\graphicspath{{/Users/telliott/Dropbox/Github-Math/figures/}}
% \begin{center} \includegraphics [scale=0.4] {gauss3.png} \end{center}

\title{Equal areas}
\date{}

\begin{document}
\maketitle
\Large

%[my-super-duper-separator]

Tycho Brahe was a Danish nobleman who made extraordinarily detailed measurements of planetary motion (by eye).  

Johannes Kepler was a German mathematician who went to work for Brahe and later inherited all of Brahe's observations.  Kepler is most famous for his three laws of planetary motion:

$\circ$ K1: The planetary orbits are ellipses with the sun at one of the foci.

$\circ$ K2: A line segment joining a planet and the Sun sweeps out equal areas during equal intervals of time.

$\circ$ K3:  The square of a planet's orbital period is proportional to the cube of the length of the semi-major axis of its orbit.

Newton explained K1 as a consequence of the $1/r^2$ dependence of the gravitational force.  I have written about this in my calculus book.

We talked about a derivation of K3 using the assumption of spherical orbits in a previous chapter.  (Some orbits are nearly spherical).  

In this chapter we'll discuss how Newton made a geometric derivation of K2:  equal areas in equal times.  It depends on some simple geometry and the idea that the force of gravity is directed at the sun.

\begin{center} \includegraphics [scale=0.4] {newton_area.png} \end{center}

Diagram the sun $S$ and a planet at $A$.  

Imagine that the force toward the sun is applied discretely.  That is, during a small interval $\Delta t$, the planet travels from $A$ to $B$ at constant velocity and if undisturbed, would travel to $C$ in the next unit of time.

In the absence of a force, the velocity would be constant and the length of $AB$ the same as that of $BC$, and then since $AB$ is on the same line as $BC$, the area of $\triangle ABS$ is equal to the area of $\triangle BCS$.  

\emph{Proof.}

Draw the vertical line from $S$ to the line containing $ABC$.  The area of either triangle is one-half the length of that altitude times the distance, either $AB$ or $BC$.  The principle is illustrated in the next figure.

\begin{center} \includegraphics [scale=0.5] {triangles_parallel.png} \end{center}

Given two parallel lines separated by a distance $h$, pick two points on one line separated by a distance $d$ and \emph{any} point on the other line.  The triangles drawn using those points will all have equal area, namely $(1/2)dh$.

$\square$

Now, suppose the force is applied at $B$ \emph{toward the sun} along $EBS$.  
\begin{center} \includegraphics [scale=0.4] {newton_area.png} \end{center}

As a result, the trajectory $BC$ is modified by the change in velocity resulting from application of the force toward the sun. The part of the path resulting from the change in velocity is the velocity resulting from the application of the force of gravity, times $\Delta t$.  

Call that length $CD$ and add it to $BC$ to give the actual trajectory, $BD$.  

$CD$ is parallel to $SBE$.  Therefore, every point on $CD$ has an altitude with respect to $SBE$ of the same length.  So any point on $CD$ can be used to draw a triangle with the same base $SB$ and the result will have the equal area no matter which point is chosen.

In particular, the area of $\triangle BDS$ is equal to the area of $\triangle BCS$, which was found earlier to be equal to the area of $\triangle ABS$.  Since the two triangles from the actual motion have the same area, the area is constant.



\end{document}  