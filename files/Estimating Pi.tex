\documentclass[11pt, oneside]{article} 
\usepackage{geometry}
\geometry{letterpaper} 
\usepackage{graphicx}
	
\usepackage{amssymb}
\usepackage{amsmath}
\usepackage{parskip}
\usepackage{color}
\usepackage{hyperref}

\graphicspath{{/Users/telliott/Dropbox/Github-math/figures/}}
% \begin{center} \includegraphics [scale=0.4] {gauss3.png} \end{center}

\title{Introduction}
\date{}

\begin{document}
\maketitle
\Large

%[my-super-duper-separator]

Here is a short Python program that carries out the estimation process which Archimedes invented to find the value of $\pi$.  It computes the cotangent (the inverse tangent) of the half-angle by adding the cotangent and the cosecant (the inverse sine) of the parent angle.

\begin{verbatim}
from math import sin, tan, radians, sqrt

def f(csc,cot):
    Cot = csc + cot
    return sqrt(1 + Cot**2),Cot
    
phi = radians(60)
x = 1/sin(phi)
y = 1/tan(phi)
n = 3

for i in range(10):
    x,y = f(x,y)
    n *= 2
    print("%4d: %3.8f" % (n,round(1/x*n,8)))
}
\end{verbatim}

This algorithm uses an inscribed polygon in a circle of diameter $1$.  When multiplied by the number of sides of the polygon, we get an estimate for $\pi$.  I started with $60^{\circ}$ so it would print the result for a hexagon after round one.

This estimate bounds $\pi$ from below.  Here is the output:
\begin{center} \includegraphics [scale=0.8] {estimate_pi.png} \end{center}

A similar approach with the tangent bounds $\pi$ from above.  Archimedes stopped at $96$, after four rounds of doubling a hexagon.

Of course, Archimedes used fractions.  He gave this part of his result as $\pi > 223/71$.  That's about $3.140845$.  The upper bound was $\pi < 22/7$, which is about $3.142857$.

Ptolemy used $377/120 = 3.141666...$ as an estimate for $\pi$.  

In the 5th century AD, Zu Chongzhi used $355/113 = 3.1415929..$.  His computation is strictly based on Pythagoras and is rather unwieldy compared to Archimedes.  However, he carried out the calculation to nine rounds, and because of that he realized that $355/113$ is a better estimate.  It is amazingly accurate.

The answer as to how Ptolemy missed it seems to be that he didn't know the true value of $\pi$ well enough to believe that $355/113$ was any better than $377/120$.  The other possibility is that he was led to his value by some logic, and didn't check nearby values systematically.

I wrote about that here:

\url{https://telliott99.blogspot.com/2021/04/pi-again.html}

\end{document}