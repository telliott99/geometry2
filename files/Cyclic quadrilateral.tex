\documentclass[11pt, oneside]{article} 
\usepackage{geometry}
\geometry{letterpaper} 
\usepackage{graphicx}
	
\usepackage{amssymb}
\usepackage{amsmath}
\usepackage{parskip}
\usepackage{color}
\usepackage{hyperref}

\graphicspath{{/Users/telliott/Github-Math/figures/}}
% \begin{center} \includegraphics [scale=0.4] {gauss3.png} \end{center}

\title{Arcs of a circle}
\date{}

\begin{document}
\maketitle
\Large

%[my-super-duper-separator]

\subsection*{theorem on cyclic quadrilaterals}

In this chapter we look at quadrilaterals where the fourth vertex is constrained to lie on the same circumcircle as the other three.  We showed this elementary theorem earlier:

\begin{center} \includegraphics [scale=0.4] {circles_4.png} \end{center}

$\bullet$ \ For \emph{any} quadrilateral whose four vertices lie on a circle, the opposing angles are supplementary (they sum to $180^\circ$).

\emph{Proof}.

Together, opposing angles exactly sweep out the whole arc of the circle.

$\square$


\subsection*{Brahmagupta's theorem}

This is a theorem credited to Brahmagupta.

\url{https://en.wikipedia.org/wiki/Brahmagupta_theorem}

\begin{center} \includegraphics [scale=0.35] {bg1.png} \end{center}

In this example, we have a special cyclic quadrilateral in which the diagonals are perpendicular.  Drop the altitude to any side, say $QP$ and extend it to meet the other side $MN$.  Then it bisects that side so $MS = SN$.

\emph{Proof}.

This is fairly straightforward.  Mark all the angles that are equal because they intersect equal arcs.  

We notice that blue and magenta are complementary.

\begin{center} \includegraphics [scale=0.35] {bg2.png} \end{center}

So then, because $SR \perp QP$, we can fill in some more dots and then use vertical angles to show that two triangles are isosceles.

\begin{center} \includegraphics [scale=0.35] {bg3.png} \end{center}

From this, we conclude that $MS = OS = SN$.

$\square$

\subsection*{problem}

Arcs of a circle often simplify problems.  Draw the altitudes and orthocenter of a triangle and draw its circumcircle.

\begin{center} \includegraphics [scale=0.35] {Posamentier1_7.png} \end{center}

Posamentier gives this relationship.  We claim that $HD = DG$.

\emph{Proof}.

It seems clear that this must be a consequence of a congruence:  it looks like $\triangle BCH \cong \triangle BCG$.  How to prove that?

By looking at arcs subtended, we can get relationships for the new angles in $\triangle BCG$, and by similar triangles we can get the others shown:
\begin{center} \includegraphics [scale=0.35] {Posamentier1_7a.png} \end{center}

We have congruent triangles by ASA, the component right triangles are congruent for the same reason.  Or just say that the altitudes of congruent triangles are also equal.

$\square$

\subsection*{Van Schooten's theorem}

This is given as a problem by Surowski (1.3.6).

Given an equilateral triangle $ABC$ draw its circumcircle.

Draw an arbitrary line segment from vertex $A$ through side $BC$ to meet the circle at $M$.  Prove that $AM = BM + MC$.

This is easy to prove as a special case of Ptolemy's theorem, which is coming.  Nevertheless, we follow the hint.
\begin{center} \includegraphics [scale=0.4] {Van_Schooten.png} \end{center}

Extend $MC$ so that $AM = MD$.  Connect $AD$.

\emph{Proof}.

Since $\triangle ABC$ is equilateral, we mark each vertex with a red dot.  

$\angle AMC$ and $\angle AMB$ are both peripheral and both are subtended by the same arc as an angle of $\triangle ABC$, hence we place two more red dots (middle).

We have drawn $AM = MD$, so $\triangle AMD$ is isosceles.  But, since the vertex $\angle AMD$ is equal to an angle that is a vertex of an equilateral triangle, $\triangle AMD$ is equilateral.  We place one red dot at $D$ and wait a bit for the other vertex.

The overlapping $\angle BAC$ and $\angle MAD$ are both vertices of equilateral triangles, so they are equal.  

Therefore, the non-overlapping parts are also equal (marked with blue dots).  We place two magenta dots at angles that are subtended by arc $MC$.

So now, we know that certainly $\triangle BAM \sim \triangle ACD$.  

But also, $AM = AD$ because $\triangle AMD$ is equilateral.  So we have ASA and $\triangle BAM \cong \triangle ACD$.

By congruent triangles the sides $BM$ and $CD$ are equal.  Given $AM = MD$, so
\[ AM = MD = MC + CD \]
\[ = MC + BM = BM + MC \]

$\square$
\begin{center} \includegraphics [scale=0.4] {Van_Schooten.png} \end{center}

\newpage

\subsection*{converse}
Suppose we are given that the angles $A$ and $C$ are supplementary, and that points $A,B,$ and $C$ lie on the circle as drawn.

Then $D$ also lies on the circle.

\begin{center} \includegraphics [scale=0.22] {Cyclic_quad_converse_b.png} \end{center}

We proved this previously \hyperref[sec:equal_angle_on_circle_contradiction]{\textbf{here}}.

\end{document}