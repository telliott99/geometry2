\documentclass[11pt, oneside]{article} 
\usepackage{geometry}
\geometry{letterpaper} 
\usepackage{graphicx}
	
\usepackage{amssymb}
\usepackage{amsmath}
\usepackage{parskip}
\usepackage{color}
\usepackage{hyperref}

\graphicspath{{/Users/telliott/Dropbox/Github-Math/figures/}}

\title{Hexagons}
\date{}

\begin{document}
\maketitle
\Large

%[my-super-duper-separator]

A hexagon can be composed of six equilateral triangles.  In an equilateral triangle, the three angles are equal and by the angle sum theorem, their measure is $\pi/3$.  This matches the requirement for a full $6 \cdot \pi/3 = 2 \pi$ angular measure around a point.

\begin{center} \includegraphics [scale=0.4] {hexagon.png} \end{center}

We simply note numerical verification of the exterior angle theorem:

\begin{center} \includegraphics [scale=0.4] {hexagon2.png} \end{center}

\[ \theta = \phi + \phi \]

There is a different theorem, why don't we call it the

\subsection*{External angle sum theorem}

\label{sec:exterior_angle_sum_theorem}

to distinguish the angle from the exterior angle of a triangle, used above.

Imagine walking all the way around a polygon, to do so we must make $n$ turns.  What is the measure of the angle at each turn, and what is the total angle turned?

It's tricky, because the diagram we have drawn above is not the right one to use.  We need this:

\begin{center} \includegraphics [scale=0.4] {hexagon3.png} \end{center}

Let's recap what we know for the smaller polygons:

$\circ$ \ triangle:  $3 \cdot 2 \pi/3 = 2 \pi$

$\circ$ \ rectangle:  $4 \cdot \pi/2 = 2 \pi$

$\circ$ \ pentagon:  $5 \cdot 2\pi/5 = 2 \pi$

$\circ$ \ hexagon:  $6 \cdot \pi/3 = 2 \pi$

The sum of the external angles is seen to be $2 \pi$, which makes sense.  We simply turn through $2 \pi = 360$ degrees total.  Therefore each angle is $2 \pi / n$.

\subsection*{approximation for pi}

\begin{center} \includegraphics [scale=0.4] {hexagon.png} \end{center}

We can use the hexagon to get two approximations, which are lower bounds, for the value of $\pi$.  Suppose the hexagon is composed of equilateral triangles with sides of unit length.

Then the perimeter is $6$ and the diameter is $2$ and their ratio is $3$.

Each triangle has a base of length $1$ and a height of length $\sqrt{3}/2$ and an area of  $\sqrt{3}/4$.  The total area is $6 \cdot \sqrt{3}/4 = 3 \sqrt{3}/2 \approx 2.6$, which is not especially close!

\subsection*{area of an $n$-gon}

A standard examination problem from 1900 might be the following:  ``The radius of a circle is 2.  What is the area of a regular inscribed dodecagon?"

We will use some basic trigonometry for this.

Start by making a diagram like the one in the left panel, below.

\begin{center} \includegraphics [scale=0.4] {area_problem2.png} \end{center}

There are $n$ sides to the polygon, so the central angle for the sector encompassing one side is $\theta = 2\pi/n$.  $\phi$ is one-half that, or $\pi/n$.

We will compute the area of the isosceles triangle whose one vertex is at the center of the circle with angle $\theta = 2 \phi$.  First, half the base is
\[ \frac{b}{2} = r \sin \phi \]
so
\[ b = 2r \sin \phi \]
and the height is
\[ h = r \cos \phi \]
so
\[ A_{\triangle} = \frac{1}{2} 2r^2 \sin \phi \cos \phi = r^2 \sin \phi \cos \phi \]

The whole polygon contains $n$ such triangles so its area is
\[ A = nr^2 \sin \phi \cos \phi \]
\[ = nr^2 \sin \pi/n \cos \pi/n \]

Compute the size of a square inscribed into a unit circle.  $n = 4$ so
\[ A = 4 \sin \pi/4 \cos \pi/4 \]
\[ = 4 \frac{\sqrt{2}}{2} \ \frac{\sqrt{2}}{2} = 2 \]

Check this (right panel, above).  We have four triangles with height and base both equal to $1$ so the total area is $4 \cdot 1/2 = 2$.  Alternatively, compute $s = 2 \cdot \sqrt{2}/2 = \sqrt{2}$ so the area is $s^2 = 2$.

The assigned problem had $r = 2$ and a dodecagon ($n = 12$), so $\theta = \pi/6$ and $\phi = \pi/12$).  The formula was
\[ A = nr^2 \sin \phi \cos \phi \]

We need the sine and cosine of this angle.  The way to do that is to use what are called the half-angle formulas with $\theta = \pi/6$, whose sine and cosine are easy.  That's real trigonometry, so we won't go through it.  However, it makes a nice exam problem because it turns out that for $\phi = \pi/12$, going through some slightly tricky algebra:

\[ \sin \phi \cos \phi = \frac{1}{4} \]

and after multiplication by $n=12$ we get $3r^2$.

A regular hexagon in a unit circle is easy.  Then $\phi = \pi/6$  The area is
\[ A = 6 \cdot \frac{1}{2} \ \frac{\sqrt{3}}{2} = 3 \frac{\sqrt{3}}{2} \]

(We have used the sine and cosine of $\pi/6$ from \hyperref[sec:sine_and_cosine]{\textbf{here}}).

We check this as follows:  a regular hexagon is composed of six equilateral triangles with unit sides.  The area of each one is $1/2 \cdot 1 \cdot \sqrt{3}/2$.  That's correct.


\end{document}