\documentclass[11pt, oneside]{article} 
\usepackage{geometry}
\geometry{letterpaper} 
\usepackage{graphicx}
	
\usepackage{amssymb}
\usepackage{amsmath}
\usepackage{parskip}
\usepackage{color}
\usepackage{hyperref}

\graphicspath{{/Users/telliott/Github-Math/figures/}}
% \begin{center} \includegraphics [scale=0.4] {gauss3.png} \end{center}

\title{Cycloid}
\date{}

\begin{document}
\maketitle
\Large

%[my-super-duper-separator]

Imagine a bicycle tire marked at a particular point on the rim, say with fluorescent paint or a small light.  Time starts at $t = 0$ with that point $P$ in contact with the $x$ axis at $(0,0)$.  Then the bike rolls to our right.  As the tire rotates the fixed point $P$ on the rim traces a curve
\begin{center} \includegraphics [scale=0.6] {cycloid.png} \end{center}

\subsection*{Roberval}
The cycloid curve was not known to the ancients, as far as our sources have anything to say.  The first mention is around 1500, and by the time of Galileo a century later, it was something he thought about quite a bit and encouraged others to study.  

Galileo is said to have used Archimedes' method of cutting out shapes and weighing them to show that the area was approximately $3$ times that of the generating circle.  But he was apparently not prepared for it to be \emph{exactly} equal, so was skeptical.

The cycloid became very popular in the decades before Newton.  Roberval came up with a clever application of Cavlieri's principle to find the area under the curve.

The great idea is to draw a second curve, called the \emph{companion curve}, by sliding values derived from a half-circle to add them to the $x$-values on the cycloid curve, as shown.  We can imagine sliding them from the left, as above, but it works also to slide them from the right.

The areas marked by horizontal lines are equal, by Cavalieri's principle.  For example $EF = PQ$.  But that total extra area is just $\pi a^2 / 2$, one-half of the circle.

\url{https://maa.org/sites/default/files/pdf/cmj_ftp/CMJ/January%202010/3%20Articles/3%20Martin/08-170.pdf}

\begin{center} \includegraphics [scale=0.2] {cycloid_comp.png} \end{center}

Notice that the companion curve $AQD$ divides the rectangular area $ACDB$ in half, by symmetry.  Each piece at a distance $h$ from the top is matched by a piece of equal length at a distance $h$ from the bottom, by the vertical symmetry of the half circle.

The rectangle has width $\pi a$ and height $2a$, so its area is $2 \pi a^2$, and one-half that is $\pi a^2$.

The area under the cycloid curve is then three-quarters of the total, which is $(3/2) \pi a^2$, and the area under one complete lobe is twice that or $3\pi a^2$.

\subsection*{slope}
Descartes gave the slope of the tangent line to the cycloid by the following construction:
\begin{center} \includegraphics [scale=0.2] {cycloid_slope.png} \end{center}
Given $P$ on the cycloid, draw $PE$ horizontally to find $E$ on the generating circle, then draw $EC$ and also $PQ$ parallel to $EC$.  The tangent is perpendicular to those two line segments.

I like this construction a lot because it reminds me of what Roberval did with the parabola.  For any $t$, the distance moved horizontally is equal to the distance along the rim of the generating circle.

\begin{center} \includegraphics [scale=0.2] {cycloid8.png} \end{center}

Indeed, in Roberval's view of the curve (see Martin)
``“Let the diameter AB of the circle AGB move along the tangent AC, always remaining parallel to its original position until it takes the position C D, and let AC be equal to the semicircle AGB. At the same time, let the point A move on the semicircle AGB, in such a way that the speed of AB along AC may be equal to the speed of A along the semicircle AGB.''

Therefore, since no slippage occurs, movement along the tangent should be in the direction of the sum of the two vectors.  The tangent is the average of the tangent to the circle at $E$ and the horizontal.  

Since the latter is zero, the result is one-half of the former.
\begin{center} \includegraphics [scale=0.2] {cycloid_slope2.png} \end{center}
We are given that $PE$ is horizontal, so it is parallel to $QC$.  Also, $PQ \parallel EC$.  Therefore, $PECQ$ is a parallelogram.  $\angle DEC$ is right.  Angles marked with blue and magenta dots are complementary.

$\angle \theta$ at the circle's origin is supplementary to two magenta dots, so $\theta/2$ is complementary to one.  That is, $\theta/2$ is complementary to the angle of the tangent, so the tangent's slope is the cotangent of $\theta/2$.

\subsection*{arc length}

According to John Martin, it was Descartes who had the idea of approximating the cycloid curve as the (bumpy) curve generated by a polygon.

\url{https://maa.org/sites/default/files/pdf/cmj_ftp/CMJ/January%202010/3%20Articles/3%20Martin/08-170.pdf}

Here is a frame from an animation
\begin{center} \includegraphics [scale=0.3] {cycloid_arc1.png} \end{center}

The Youtube video gives a derivation for the length of the curve, first solved by Christopher Wren, although the proof given is not the same as his.

\url{https://www.youtube.com/watch?v=3yK2tJZR3Js}

For a polygon of $n$ sides, we will have n-1 circular arcs.  The arc generated by rotating a polygon around one of its vertices subtends the same angle for each vertex.  \emph{Proof}.  The angle of rotation is just the exterior angle of the polygon.  $\square$

What does change is the radius of each arc element.  The first one is the side length, but the second one is longer.  For example, with the square it is the diagonal of the square.  And in general, the length of the arm that sweeps out the curve is the length of a chord that connects $k = 1, 2, 3 \dots$ faces.

\begin{center} \includegraphics [scale=0.25] {cycloid_arc2.png} \end{center}

Notice that the angle subtended by any face (or multiple adjacent faces) of a polygon is the same regardless of the position of the face with respect to the vertex.  \emph{Proof}.  The central angle for any two adjacent vertices is the same.  So the peripheral angles are all equal, by the inscribed angle theorem.  $\square$

But these arms are also chords!  And that is how we will get their lengths.  A single side is the chord for a central angle $2 \pi/n$.  Two sides for twice that.  $k$ sides for $2k \pi/n$.

Recall that if $t$ is the central angle subtended by any chord, then the chord length is twice the sine of the half-angle, times the radius of the circle.
\[ c = 2a \sin \frac{t}{2} \]

So if we divide into $n$ sides and we are on the $k$th arc with the chord of $k$ sides as the arm
\[ c_k = 2a \sin \frac{k \pi}{n} \]

The chord length is also the radius of the circular sector.  We get the arc length as the radius times the angle.  The $k$th side has arc length

\[ 2a \sin \frac{k \pi}{n} \cdot \frac{2 \pi}{n} \]

Sum them all
\[ \sum_1^{n-1}  2a \sin \frac{k \pi}{n} \cdot \frac{2 \pi}{n} \]

We can add an extra term at the bottom and the top (because $\sin 0 = \sin \pi = 0$:
\[ \sum_0^{n}  2a \sin \frac{k\pi}{n} \cdot \frac{2 \pi}{n} \]
\[ \frac{4 \pi}{n}  \sum_{k=0}^{n}  \sin \frac{k\pi}{n}  \]
Now we need to find out what happens to $\sin \frac{k\pi}{n}$ as $n \rightarrow \infty$.

\subsection*{Lagrange}
We use Lagrange's trigonometric identity.  (We follow the video even though I found another way to do it, and a puzzle. --- see the next section).
\begin{center} \includegraphics [scale=0.25] {cycloid_arc3.png} \end{center}

\[ \sum_{k=0}^n \sin k \theta = \frac{\sin \frac{n \theta}{2} \cdot \sin (\frac{n \theta}{2} + \frac{\theta}{2})}{\sin \frac{\theta}{2}} \]

The argument to the sine function $k \theta$ is $\frac{\pi}{n}$.  That is, $\theta = \frac{\pi}{n}$.

The first term in the numerator is
\[ \sin \frac{n \theta}{2} = \sin \frac{n \pi}{2n} = \sin \frac{\pi}{2}= 1 \]
And the second is
\[ \sin (\frac{n \theta}{2} + \frac{\theta}{2}) = \sin(\frac{\pi}{2} + \frac{\pi}{2n})   \] 
Recall that $n \rightarrow \infty$.  Hence this expression is also equal to $1$.

For the denominator, we use the small angle approximation, since as $n$ becomes large, $\frac{\theta}{2} = \frac{\pi}{n}$ becomes very small.  

So $\frac{\theta}{2} \approx \sin \frac{\theta}{2}$ and then 
\[ \sum_0^{n}  \sin k \pi/n = \frac{1}{\frac{\pi}{2n}} = \frac{2n}{\pi} \]

Multiply by what was in front to obtain the answer $8a$ (for a circle or set of polygons with $n$ sides and radius $a$).

\subsection*{puzzle with the proof}
However, there is a puzzle with the proof as presented.  According to wikipedia

\url{https://en.wikipedia.org/wiki/List_of_trigonometric_identities#Lagrange's_trigonometric_identities}

The well-known formula for the sine is
\[ \sum_{k=0}^n \sin k \theta = \frac{\cos \frac{1}{2} \theta - \cos ((n + \frac{1}{2}) \theta)}{2 \sin \frac{1}{2} \theta} \]

Derivation:
\[ \cos (A + B) = \cos A \cos B - \sin A \sin B \]
\[ \cos (A - B) = \cos A \cos B + \sin A \sin B \]
Subtract the first from the second:
\[ \cos (A - B) - \cos (A + B) = 2 \sin A \sin B \]
\[ 2 \sin B \sum \sin A = \sum \cos (A - B) - \cos (A + B) \]

If $B = \frac{1}{2}$ and $A$ consists of integral $\theta$ as in $A = k \theta$, then the series will consist of terms like $\cos (k - \frac{1}{2}) \theta$ followed by $- \cos ((k + 1 + \frac{1}{2}) \theta)$.  These cancel each other, forming a \emph{telescoping series}.

The only terms that survive come from the first and last:
\[ \cos \frac{1}{2} \theta - \cos (n+\frac{1}{2}) \theta \]
(Actually for this derivation it is more convenient to start the sum from $k=1$).

We have $\theta = \frac{\pi}{n}$.  As $n \rightarrow \infty$, the first term is $\cos 0 = 1$ and the second term is $- \cos (\pi + 0) = 1$.

The sum is $2$ which cancels the $2$ on the bottom.  The rest of the denominator is
\[ \sin \frac{1}{2} \theta \approx \frac{\pi}{2n}  \]
which inverts and multiplies to give exactly the same answer as before.

One can also find online proofs using Euler's identity.

\url{https://math.stackexchange.com/questions/1349466/calculating-sum-k-0n-sink-theta} 

which gives the first identity, where the terms are all sines, the right hand side has a multiplication, and there is no factor of $2$.

I have written out a full version of both of these derivations separately.  

\url{https://github.com/telliott99/short_takes/blob/master/Lagrange%20trig.pdf}

I also found a proof that the two formulas are equivalent, which is included.

\subsection*{standard parametric approach}

We want to find equations that give the point $P$ as a function of time.  We will parametrize the curve, yielding parametric equations $x(t)$, $y(t)$.

The diagram below shows the angle through which the wheel has turned as $\theta$, but we will use $t$ for $\theta$ here.  

\begin{center} \includegraphics [scale=0.5] {cycloid2.png} \end{center}
The $x$ displacement of the vertical straight down from the center of the tire is just $at$, where $a$ is the radius of the wheel, it is equal to the arc on the circumference of the wheel from the point which is currently in contact with the ground, going around up to $P$.

It is reasonably easy to derive the desired parametric equations, using vectors, especially once you know the answer.  For $x$, we have the vector that goes from $(0,0)$ to the contact point with the ground.  As indicated in the figure, that is $at$.  

We need to subtract the distance $a \ sin\ t$ from that.  Basically the rationale is that the motion is a standard parametric circle which has been rotated by $90$ degrees clockwise and then inverted.  The rotation changes cosine to sine, and the inversion brings the subtraction.

\begin{center} \includegraphics [scale=0.35] {cycloid3.png} \end{center}

It's easier to see for $t < \pi/2$, but it is true always.  Check some other values of $t$ like $\pi$ or $3\pi/2$ to confirm.  This is the usual circular motion.

For $y$, we have a constant factor of $a$ above the $x$ axis, then the additional displacement is $-a \ cos \ t$.  So for $t=0$ we have the additional displacement is -a (we were on the ground), for $t=\pi/2$ it is zero, and for $t=\pi$ it is plus $a$ for a total of $2a$.

The parametric equations are then
\[ x(t) = at - a \sin t \]
\[ y(t) = a - a \cos t \]

Here is an animation from Desmos, if you want to set it up and play with it.
\begin{center} \includegraphics [scale=0.18] {cycloid6.png} \end{center}

It is almost, but not quite, a semi-ellipse.
\begin{center} \includegraphics [scale=0.18] {cycloid7.png} \end{center}

If you look closely at the animation, we had
\[ \sin (at + \pi) + at = at - \sin at \]
\[ \cos (at + \pi) + 1 = 1 - \cos at \]
where we have used the addition formulas
\[ \sin A+B = \sin A \cos B + \sin B \cos A \]
\[ \cos A+B = \cos A \cos B - \sin A \sin B \]

So we see that the Desmos formulas are actually the same as what we've written above (and used in the second screenshot, regarding the semi-ellipse.

Once we know some calculus, we can get various expressions like the slope of the curve.  For example, the derivatives are:
\[ x'(t) = a - a \cos t \]
\[ y'(t) = a \sin t  \]

We can get the slope of the tangent line, by simple division:
\[ y' = \frac{dy}{dx} = \frac{a \sin t }{a - a \cos t} \]
\[ = \frac{\sin t}{1 - \cos t} \]

Simmons uses half-angle formula to make something simpler, like so:
\[ \sin t = 2 \sin \frac{t}{2} \cos \frac{t}{2} \]
\[ \cos t = \cos^2 \frac{t}{2} - \sin^2 \frac{t}{2} \]
\[ = 1 - 2 \sin^2 \frac{t}{2} \]
So the ratio is
\[ y' = \frac{ 2 \sin \frac{t}{2} \cos \frac{t}{2}}{2 \sin^2 \frac{t}{2}} = \cot \frac{t}{2} \]
We saw this before, when we looked at the geometry of the tangent.

\subsection*{aside about Archimedes}

It struck me that $\cot t/2$ is one of the terms in the relationship Archimedes used in approximating $\pi$.  There we had
\[ \cot 2 \theta + \csc 2 \theta = \cot \theta \]
\[ \cot t + \csc t = \cot t/2 \]
So somehow, if what we have above is correct
\[ \frac{\sin t}{1 - \cos t} = \cot t/2 = \cot t + \csc t \]
Factor the right-hand side
\[  = \frac{1}{\sin t} (\cos t + 1) \]
Then
\[ \sin^2 t = (1 - \cos t)(\cos t + 1) \]
\[ = 1 - \cos^2 t \]
and one more step gives our favorite identity.

Let's see if we can figure out a parametric equation for the companion curve.  $x(t)$ for the cycloid was $x(t) = at - a \sin t$.  The companion curve gets an additional length in the $x$-direction of $a \sin t$.  So $x(t) = at$.  That was easy!

If we want $y = f(x)$ we must do
\[ y = a - a \cos t \]
and then substitute $t = x/a$ so
\[ y = a(1 - \cos x/a) \]

which goes like the cosine of a scaled version of $x$, shifted up
\begin{center} \includegraphics [scale=0.2] {cycloid5.png} \end{center}

Since the latter is zero, the result is one-half of the former.

Referring to the figure below, and repeating what we said before, $\angle \theta$ at the circle's origin is supplementary to two magenta dots, so $\theta/2$ is complementary to one.  That is, $\theta/2$ is complementary to the angle of the tangent, so the tangent's slope is the cotangent of $\theta/2$.

\begin{center} \includegraphics [scale=0.2] {cycloid_slope2.png} \end{center}

Let us also take the origin of the circle temporarily as the origin of coordinates $(0,0)$.  Then let point $E$ be
\[ x' = -a \sin t  \]
\[ y' = a -a \cos t \]
Since $C = (0,-a)$ we have that the slope of $CE \parallel PQ$ is
\[  \frac{\Delta y}{\Delta x} = \frac{(a - a \cos t) - a}{(-a \sin t) - 0} = \frac{\cos t - 1}{\sin t} \]
The slope of the tangent is the negative inverse
\[ \frac{\sin t}{1 - \cos t} \]
which matches.  We can also show that $ED$ has the same slope as the tangent, since it makes a right angle with $CE$.


\end{document}